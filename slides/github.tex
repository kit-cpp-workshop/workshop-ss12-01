\section{GitHub}

\begin{frame}{Was ist GitHub?}
	\begin{itemize}
		\item \url{http://www.github.com}
		\item Hoster für Software-Projekte, insbesondere für Git-Repositories
		\item zahlreiche Funktionen zum Zusammenarbeiten
		\item kostenlos für open-source-Projekte
		\item im Browser lassen sich die Dateien des repos betrachten und verändern
	\end{itemize}
\end{frame}

\begin{frame}[fragile]{Wie nutzen wir GitHub?}
	\small{
	Derzeit (pair programming) ist der Workflow:
	
	\begin{enumerate}
		\item Wir hosten die Materialien zu einem Workshop in einem \emph{repo}.
		\item Du forkst das repo in deinen GitHub-Account.
		\item Dann clonest du dir \emph{deinen Fork} auf deinen Rechner (\verb|git clone|).
		\item Jetzt kannst du mit deiner lokalen Kopie arbeiten, z.B. die Aufgaben lösen.
		\item Nutze Git auch lokal zur Versionsverwaltung! Committe häufig!
		\item Nach dem Lösen \emph{aller Aufgaben} und Committen: pushen (damit landen die Änderungen in deinem Fork bei GitHub).
		\item Erstelle auf GitHub einen \emph{pull request} und nenne deinen Betreuer im Text.
		\item Ein Betreuer wird über deine Lösung schauen, diese kommentieren \emph{und den pull request ablehnen}.
	\end{enumerate}
	}
\end{frame}
