%% LaTeX-Beamer template for KIT design
%% by Erik Burger, Christian Hammer
%% title picture by Klaus Krogmann
%%
%% version 2.1
%%
%% mostly compatible to KIT corporate design v2.0
%% http://intranet.kit.edu/gestaltungsrichtlinien.php
%%
%% Problems, bugs and comments to
%% burger@kit.edu

\documentclass[18pt]{beamer}

\usepackage[utf8]{inputenc}
\usepackage[babel,german=quotes]{csquotes}
\usepackage{graphicx}
\usepackage{caption}
\usepackage{subfig}
\usepackage[right]{eurosym}
\usepackage{listings}

%% SLIDE FORMAT

% use 'beamerthemekit' for standard 4:3 ratio
% for widescreen slides (16:9), use 'beamerthemekitwide'

\usepackage{templates/beamerthemekit}
% \usepackage{templates/beamerthemekitwide}

%% TITLE PICTURE

% if a custom picture is to be used on the title page, copy it into the 'logos'
% directory, in the line below, replace 'mypicture' with the 
% filename (without extension) and uncomment the following line
% (picture proportions: 63 : 20 for standard, 169 : 40 for wide
% *.eps format if you use latex+dvips+ps2pdf, 
% *.jpg/*.png/*.pdf if you use pdflatex)

\titleimage{title}

%% TITLE LOGO

% for a custom logo on the front page, copy your file into the 'logos'
% directory, insert the filename in the line below and uncomment it

\titlelogo{titlelogo}

% (*.eps format if you use latex+dvips+ps2pdf,
% *.jpg/*.png/*.pdf if you use pdflatex)

%% TikZ INTEGRATION

% use these packages for PCM symbols and UML classes
% \usepackage{templates/tikzkit}
% \usepackage{templates/tikzuml}

% the presentation starts here

\title[C++ Workshop]{C++ Workshop}
\subtitle{1. Block, 27.04.2012\\ orga-Kram}
\author{Jingfan Ye, Jan Homberg, Robert Schneider}

\institute{}

\begin{document}

% change the following line to "ngerman" for German style date and logos
\selectlanguage{ngerman}

%title page
\begin{frame}
\titlepage
\end{frame}

%table of contents
\begin{frame}{Gliederung}
\tableofcontents
\end{frame}

\section{Organisatorisches}


\subsection{Platzzahl}

\begin{frame}{Begrenzung der Platzzahl}
	\begin{block}{Warum eine Begrenzung?}
		\begin{itemize}
			\item Raumgröße
			\item Betreuung, Gruppengröße
		\end{itemize}
	\end{block}
	\ \\
	\pause
	\ \\
	\begin{block}{Platzzahl}
		\begin{itemize}
			\item Max. 24 Plätze, Betreuer zusätzlich
			\item Warteschlange (wie bei Sprachkurs)
			\item online-Teilnahme möglich, Fragen per Mail an FOO@BAR.COM \note{einer muss deligieren}
		\end{itemize}
	\end{block}
\end{frame}

\begin{frame}{Wie bekomme ich einen Platz?}
	Nach folgenden Kriterien werden die Plätze verteilt:
	\begin{enumerate}
		\item Anwesenheit heute (27.4.) oder Entschuldigung
		\item Mindestvoraussetzung erfüllen (\enquote{Programmieren für Physiker}, siehe git repo GIT REPO!!!!!!!)
		\item Aktive Teilnahme und Interesse (der \emph{Willen}, teilzunehmen und das durchzuziehen)
		\item first-come, first-served (Wer zuerst kommt, malt zuerst.)
	\end{enumerate}
	
	Nachrücken ist möglich!
\end{frame}



\subsection{Organisatorisches}

\begin{frame}{Kontakt, Kooperation}
	\begin{description}
		\item[Mailing-List] cpp-workshop@lists.kit.edu
		\item[github]	\url{www.github.com}, Organization kit-cpp-workshop, Anmeldung via Christian
	\end{description}
	\ \\
	
	Namen und E-Mail-Adressen:
	\begin{table}
		\begin{tabular}{l|l}
			Christian Käser	&	Christian.Kaeser@student.kit.edu	\\
			\hline
			Markus Jung		&	Markus.Jung@stud.uni-karlsruhe.de	\\
			\hline
			Matthias Blaicher	&	matthias@blaicher.com	\\
			\hline
			Robert Schneider	&	Robert.Schneider3@student.kit.edu	\\
			\hline
			Sven Brauch	&	SvenBrauch@googlemail.com	\\
		\end{tabular}
	\end{table}
\end{frame}

\begin{frame}{Der Workshop}
	\begin{block}{Termin}
		\begin{itemize}
			\item jeden Freitag in der Vorlesungszeit
			\item 15:45-17:15
			\item Raum: 2-0 (Physik-Hochhaus)
			\item evtl. Zusatztermine nach Absprache (z.B. cmake)
		\end{itemize}
	\end{block}
	\pause
	\begin{block}{Ziele}
		\begin{itemize}
			\item Erfahrungsaustausch!
			\item Hilfe zur Selbsthilfe / Anhaltspunkte zum Weiterlernen
			\item Tiefergehenderer Einblick in die Sprache
			\item Verschiedene Gebiete von C++ (\enquote{Schweizer Taschenmesser})
			\item Kooperation, Team-Programmierung
			\item Software-Design, programming idioms, Fehler vermeiden usw.
		\end{itemize}
	\end{block}
\end{frame}

\begin{frame}{Struktur eines Workshops}
	\begin{block}{Themen}
		\begin{itemize}
			\item Richten sich nach den Interessen der Teilnehmer!
			\item Grundlegende Methodiken, Werkzeuge, ...
			\item Objektorientierung
			\item Algorithmen \& Datenstrukturen
			\item Verwendung von Bibliotheken (statisch, dynamisch gelinkt)
		\end{itemize}
	\end{block}
	\pause
	\begin{block}{Aufteilung}
		\begin{itemize}
			\item Theoretischer Teil, heute: Organisatorisches, git, eclipse
			\item Praktischer Teil: pair programming (2er-Teams) IDE + hello-world
			\item Übungen: im praktischen Teil, zu Hause fertig. Vorstellung einiger Lösungen in darauffolgender Woche
		\end{itemize}
	\end{block}
\end{frame}


\appendix
\beginbackup

\begin{frame}[allowframebreaks]{Links}
\end{frame}

\backupend

\end{document}
